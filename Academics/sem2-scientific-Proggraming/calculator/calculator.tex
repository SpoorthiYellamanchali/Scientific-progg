\let\negmedspace\undefined
\let\negthickspace\undefined
\documentclass[journal]{IEEEtran}
\usepackage[a5paper, margin=10mm, onecolumn]{geometry}
%\usepackage{lmodern} % Ensure lmodern is loaded for pdflatex
\usepackage{tfrupee} % Include tfrupee package

\setlength{\headheight}{1cm} % Set the height of the header box
\setlength{\headsep}{0mm}     % Set the distance between the header box and the top of the text

\usepackage{gvv-book}
\usepackage{gvv}
\usepackage{cite}
\usepackage{amsmath,amssymb,amsfonts,amsthm}
\usepackage{algorithmic}
\usepackage{graphicx}
\usepackage{textcomp}
\usepackage{xcolor}
\usepackage{txfonts}
\usepackage{listings}
\usepackage{enumitem}
\usepackage{mathtools}
\usepackage{gensymb}
\usepackage{comment}
\usepackage[breaklinks=true]{hyperref}
\usepackage{tkz-euclide} 
\usepackage{listings}
% \usepackage{gvv}                                        
\def\inputGnumericTable{}                                 
\usepackage[latin1]{inputenc}                                
\usepackage{color}                                            
\usepackage{array}                                            
\usepackage{longtable}                                       
\usepackage{calc}                                             
\usepackage{multirow}                                         
\usepackage{hhline}                                           
\usepackage{ifthen}                                           
\usepackage{lscape}
\begin{document}

\bibliographystyle{IEEEtran}
\vspace{3cm}

\title{Scientific Calculator}
\author{EE24BTECH11065 - Spoorthi yellamanchali
}
% \maketitle
% \newpage
% \bigskip
{\let\newpage\relax\maketitle}

\renewcommand{\thefigure}{\theenumi}
\renewcommand{\thetable}{\theenumi}
\setlength{\intextsep}{10pt} % Space between text and floats


\numberwithin{equation}{enumi}
\numberwithin{figure}{enumi}
\renewcommand{\thetable}{\theenumi}


\section{Aim}
To build a Scientific calculator.

\section{Hardware Components}
The main components used in this project are listed in Table \ref{tab:components}.
\begin{table}[h!]
\centering
\begin{tabular}{|c|c|}
\hline
\textbf{Component} & \textbf{Quantity} \\
\hline
Arduino Microcontroller & 1 \\
\hline
Non-I2C 16x2 LCD & 1 \\
\hline
Push Buttons (0-9 digits) & 10 \\
\hline
Push Buttons (+, -, *, /) & 4 \\
\hline
Push Button (Scroll for sin, cos, tan) & 1 \\
\hline
15k\textohm Resistors & 10 \\
\hline
1k\textohm Resistors & 1 \\
\hline
Breadboard & 1 \\
\hline
Jumper Wires & As required \\
\hline
\end{tabular}
\caption{List of Components}
\label{tab:components}
\end{table}


\section{Circuit Connections}
The LCD is operated in 4-bit mode to reduce the number of control pins. The key connections are given in Table \ref{tab:connections}.
\begin{table}[h!]
\centering
\begin{tabular}{|c|c|}
\hline
\textbf{LCD Pin} & \textbf{Connection} \\
\hline
Vss & GND \\
\hline
Pin 2 & 5V \\
\hline
Pin 3 & 1k\textohm resistor to GND \\
\hline
Pin 4 & Arduino Pin 2 \\
\hline
Pin 5 & Connected to all push buttons \\
\hline
Pin 6 & Arduino Pin 3 \\
\hline
Pin 15 & 1k\textohm resistor to 5V \\
\hline
Pin 16 & GND \\
\hline
\end{tabular}
\caption{LCD Pin Connections}
\label{tab:connections}
\end{table}


\section{Push buttons}
\begin{table}[h!]
    \centering
    \begin{tabular}{|c|c|}
        \hline
        \textbf{Button Function} & \textbf{Arduino Connection} \\
        \hline
        Digit 0-9 Selection & A0 (Analog Read) \\
        \hline
        Addition (+) & Digital Pin 13 \\
        \hline
        Subtraction (-) & Digital Pin 12 \\
        \hline
        Multiplication (*) & Digital Pin 11 \\
        \hline
        Division (/) & Digital Pin 10 \\
        \hline
        Open Bracket (() & Digital Pin 9 \\
        \hline
        Close Bracket ()) & A3 (Analog Read) \\
        \hline
        Logarithm (ln/log) & A4 (Analog Read) \\
        \hline
        Evaluate (Calculate/Enter) & Digital Pin 8 \\
        \hline
        Decimal Point (.) & Digital Pin 1 \\
        \hline
        Delete (Backspace) & Digital Pin 0 \\
        \hline
    \end{tabular}
    \caption{Button Connections for Scientific Calculator}
    \label{tab:button_connections}
\end{table}


\section{Analog-to-Digital Conversion (ADC) Values}
Analog-to-Digital Conversion (ADC) is a technique used to convert an analog signal into a digital value that can be processed by a microcontroller. In this project, ADC values are used to read multiple button inputs using a single analog pin. This is achieved by connecting different resistors to each button, creating a unique voltage divider network. When a button is pressed, a specific voltage is applied to the analog pin, which the microcontroller reads and maps to a corresponding digital value.

The main advantages of using ADC values for button inputs include:
\begin{itemize}
\item Reduction in the number of required input pins, optimizing microcontroller resources.
\item Simplified wiring, reducing circuit complexity and improving reliability.
\item Efficient and accurate button detection using predefined voltage thresholds.
\end{itemize}

The ADC technique allows multiple buttons to be read through a single analog pin, saving digital input pins. Arithmetic operations are handled through separate digital pins, and trigonometric functions are accessed using a single button with a scrolling method



\section{Software Implementation}
The firmware is developed in Embedded C and compiled using AVR-GCC. The main functions include:
\begin{itemize}
\item Reading ADC values for digit buttons
\item Detecting arithmetic operations using digital pins
\item Implementing a scroll method for trigonometric functions
\item Displaying input and results on the LCD
\item Performing calculations and updating the display dynamically
\end{itemize}

\section{Advantages of AVR-GCC}
AVR-GCC (GNU Compiler Collection for AVR) offers several benefits:
\begin{itemize}
\item Open-source and free, making it accessible for all developers.
\item Efficient optimization techniques that reduce program size and execution time.
\item Portability across different AVR microcontrollers.
\item Extensive library support for handling hardware interfaces such as LCD and ADC.
\item Compatibility with various development environments, including Atmel Studio and Arduino IDE.
\end{itemize}

\section{Conclusion} This project successfully demonstrates the implementation of a scientific calculator using an Arduino microcontroller. By efficiently utilizing ADC values, digital pins, and a non-I2C LCD, the design minimizes hardware complexity while maintaining functionality. The use of AVR-GCC enhances performance and flexibility, making the project a practical and cost-effective solution for embedded applications.




    
\end{document}


