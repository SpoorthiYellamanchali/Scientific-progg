\documentclass{beamer}
\mode<presentation>
\usepackage{amsmath}
\usepackage{amssymb}
%\usepackage{advdate}
\usepackage{adjustbox}
\usepackage{subcaption}
\usepackage{enumitem}
\usepackage{multicol}
\usepackage{mathtools}
\usepackage{listings}
\usepackage{url}
\def\UrlBreaks{\do\/\do-}
\usetheme{Boadilla}
\usecolortheme{lily}
\setbeamertemplate{footline}
{
  \leavevmode%
  \hbox{%
  \begin{beamercolorbox}[wd=\paperwidth,ht=2.25ex,dp=1ex,right]{author in head/foot}%
    \insertframenumber{} / \inserttotalframenumber\hspace*{2ex} 
  \end{beamercolorbox}}%
  \vskip0pt%
}
\setbeamertemplate{navigation symbols}{}

\providecommand{\nCr}[2]{\,^{#1}C_{#2}} % nCr
\providecommand{\nPr}[2]{\,^{#1}P_{#2}} % nPr
\providecommand{\mbf}{\mathbf}
\providecommand{\pr}[1]{\ensuremath{\Pr\left(#1\right)}}
\providecommand{\qfunc}[1]{\ensuremath{Q\left(#1\right)}}
\providecommand{\sbrak}[1]{\ensuremath{{}\left[#1\right]}}
\providecommand{\lsbrak}[1]{\ensuremath{{}\left[#1\right.}}
\providecommand{\rsbrak}[1]{\ensuremath{{}\left.#1\right]}}
\providecommand{\brak}[1]{\ensuremath{\left(#1\right)}}
\providecommand{\lbrak}[1]{\ensuremath{\left(#1\right.}}
\providecommand{\rbrak}[1]{\ensuremath{\left.#1\right)}}
\providecommand{\cbrak}[1]{\ensuremath{\left\{#1\right\}}}
\providecommand{\lcbrak}[1]{\ensuremath{\left\{#1\right.}}
\providecommand{\rcbrak}[1]{\ensuremath{\left.#1\right\}}}
\theoremstyle{remark}
\newtheorem{rem}{Remark}
\newcommand{\sgn}{\mathop{\mathrm{sgn}}}
\providecommand{\abs}[1]{\left\vert#1\right\vert}
\providecommand{\res}[1]{\Res\displaylimits_{#1}} 
\providecommand{\norm}[1]{\lVert#1\rVert}
\providecommand{\mtx}[1]{\mathbf{#1}}
\providecommand{\mean}[1]{E\left[ #1 \right]}
\providecommand{\fourier}{\overset{\mathcal{F}}{ \rightleftharpoons}}
%\providecommand{\hilbert}{\overset{\mathcal{H}}{ \rightleftharpoons}}
\providecommand{\system}{\overset{\mathcal{H}}{ \longleftrightarrow}}
%\newcommand{\solution}[2]{\textbf{Solution:}{#1}}
%\newcommand{\solution}{\noindent \textbf{Solution: }}
\providecommand{\dec}[2]{\ensuremath{\overset{#1}{\underset{#2}{\gtrless}}}}
\newcommand{\myvec}[1]{\ensuremath{\begin{pmatrix}#1\end{pmatrix}}}
\let\vec\mathbf

\lstset{
%language=C,
frame=single, 
breaklines=true,
columns=fullflexible
}

\numberwithin{equation}{section}

\title{NCERT-8.1.ex.5}
\author{Spoorthi Yellamanchali - EE24BTECH11065}

\date{\today} 
\begin{document}

\begin{frame}
\titlepage
\end{frame}

\begin{frame}
\frametitle{Outline}
\tableofcontents
\end{frame}

\section{Question}
\begin{frame}
\frametitle{Question}
Find the area bounded by the ellipse $\frac{x^2}{a^2}$+$\frac{y^2}{b^2}$ = 1 and the ordinates $x=0$ and $x = ae$, where, $b^2 = a^2\brak{1 - e^2}$ and $e < 1$.
\\
\end{frame}

\section{Theoretical Solution}
\begin{frame}
\frametitle{kdnwd}
The Differential Equation is of the form:
\begin{align}
 y = \pm \frac{b}{a}\sqrt{a^2 - x^2}
\end{align}
Area $A$ enclosed is equal to 
\begin{align}
    A = 2\int_0^{ae} \frac{b}{a}\sqrt{a^2 - x^2}\,dx
    \end{align}
    \begin{align}
      = \frac{2b}{2a}\left[ae\sqrt{a^2 - \brak{ae}^2} + a^2\sin^{-1}{e} \right]
      \end{align}
\begin{align}
      = ab\left[e\sqrt{1 - e^2} + \sin^{-1}{e}\right]
\end{align}
\end{frame}

\section{Solution by using Trapezoidal rule}
\subsection{Explanation and usage of Trapezoidal rule}
\begin{frame}
\frametitle{Trapezoidal rule}
In the trapezoidal rule,we calculate the area of the curve by breaking the whole area into trapeziums of small areas and adding them up.\\
Under trapezoidal rule,Area $A$ is approximated by,
\begin{align}
    J = \int_a^b f(x)\,dx \approx h\brak{\frac{1}{2}\brak{f(a) + f(x_1)} + \frac{1}{2}\brak{f(x_1) + f(x_2)} + ..........+ \frac{1}{2}\brak{f(x_{n-1})+ f(b)}}
\end{align}
\begin{align}
    J = \int_a^b f(x)\,dx \approx h\brak{\frac{1}{2}f(a) + f(x_1) + f(x_2) + ..... + f(x_{n-1} + \frac{1}{2}f(b)}
\end{align}
Where $h$ is assumed to be the distance between parallel sides of trapezium and its value is taken close to zero.\\
so the number of trapeziums the required area is broken into can be written as,
\begin{align}
    n = \frac{b - a}{h}
\end{align}
\end{frame}

\subsection{Difference equation}
\begin{frame}
\frametitle{Difference equation obtained from trapezoidal rule: }
On observing equation $\brak{0.11}$, we can see that,\\
If $A\brak{x_n}$ Onis area enclosed by the curve $f\brak{x}$, from $x=x_0$ to $x=x_n$,then,
\begin{align}
    A\brak{x_n} = A\brak{x_{n-1}} + \frac{1}{2}h\brak{f\brak{x_{n-1}} + f\brak{x_n}}
\end{align}
From the method of finite differences , we know that,
\begin{align}
    x_n = x_{n-1} + h
\end{align}
\begin{align}
    y_{n} = y_{n-1} + h\brak{\frac{dy}{dx}|_{x=x_{n-1}}}
\end{align}
On differentiating equation $\brak{as}$, and substituting the expression for $\frac{dy}{dx}$ in equation $\brak{a}$, we get the difference equation for the curve which is,
\begin{align}
     y_{n} = y_{n-1} + h\brak{\frac{-bx}{a\sqrt{a^2 - x^2}}}
\end{align}
\begin{align}
  A_{n+1}&=A_n+\frac{1}{2}h\brak{\brak{y_{n}+hy^{\prime}_n}+y_n}\\
  A_{n+1}&=A_n+\frac{1}{2}h\brak{2y_n+hy^{\prime}_n}\\
  A_{n+1}&=A_n+hy_n+\frac{1}{2}h^2y^{\prime}_n\\
\end{align}
\end{frame}

\subsection{Applying Trapezoidal rule}
\begin{frame}
\frametitle{Applying Trapezoidal rule}
In the given question, $y_n=\frac{b}{a}\sqrt{a^2-x_n^2}$ and $y^{\prime}_n= \frac{-bx_n}{\brak{a\sqrt{a^2-x_n^2}}}$\\
The general difference equation will be given by
\begin{align}
  A_{n+1}&=A_n+hy_n+\frac{1}{2}h^2y^{\prime}_n\\
  A_{n+1}&=A_n+h\brak{\frac{b}{a}\sqrt{a^2-x_n^2}}+\frac{1}{2}h^2\brak{\frac{-bx_n}{\brak{\sqrt{a^2-x_n^2}}}}\\
\end{align}
We know that,
\begin{align}
    x_0 = a\\
    x_n = ae
\end{align}
On changing different values of $h$, we will observe that,as value for $h$ approaches zero,the area calculated using trapezoidal rule approaches the area calculated theoretically.
On assuming a value fro $h$ close to zero and substituting in equation and on iterating till we reach $x_n=ae$ will return an area.\\
$\therefore$ our required area is twice the returned area.\\
\end{frame}

\subsection{Factors affecting accuracy of the trapezoidal rule: }
\begin{frame}
\frametitle{testing trapezoidal rule by an example}
let us take $a = 4$,$e = \frac{\sqrt{7}}{4}$ and observe for different $h$ values.
\begin{enumerate}
    \item  $h = 0.1$\\
    In this case we get, \\
Theoretical area is:  14.625751423656318\\
Area using trapezoidal rule:  14.870164029287624\\
\item $h = 0.01$\\
Theoretical area is:  14.625751423656318\\
Area using trapezoidal rule:  14.644880662091099\\
\item $h = 0.001$
Theoretical area is:  14.625751423656318\\
Area using trapezoidal rule:  14.626870703450807\\
\end{enumerate}
we can observe that as value of $h$ approaches zero,the area calculated using trapezoidal rule approaches the theoretical area.
\end{frame}

\subsection{Region of Convergence}
\begin{frame}
\frametitle{Region of Convergence}
Each term in $Y\brak{s}$ contributes a pole:
The term $\frac{1}{s+3}$ has a pole at $s = -3$, and the term $\frac{1}{s+2}$ has a pole at $s = -2$.\\
\textbf{Right-sided (causal):} The ROC is to the right of the rightmost pole, i.e., $ \text{Re}(s) > -2 .$\\
\textbf{Left-sided (anti-causal):} The ROC is to the left of the leftmost pole, i.e., $ \text{Re}(s) < -3 .$\\
\textbf{Two-sided:} The ROC lies between the poles, i.e., $-3 < \text{Re}(s) < -2 .$
\end{frame}

\subsection{Solution by Laplace Transform}
\begin{frame}
\frametitle{Solution by Laplace Transform}
Using the inverse Laplace transform:
\begin{align}
    \mathcal{L}^{-1}\brak{\frac{1}{s + a}} = e^{-ax}.
\end{align}
Substituting in the above equation:
\begin{align}
y\brak{x} &= \mathcal{L}^{-1}\brak{\frac{1}{s + 3}} + \mathcal{L}^{-1}\brak{\frac{1}{s + 2}} \\
y\brak{x} &= e^{-3x} + e^{-2x}.
\end{align}
The Laplace Transform simplifies linear differential equations with constant coefficients into algebraic equations in the Laplace domain.
\end{frame}

\subsection{Plotting the Curve}
\begin{frame}
\frametitle{Plotting the Curve}
To trace the curve, we calculate the slope of the tangent at a point on the curve. From \eqref{eq:1}, the slope is given by:
\begin{align}
\frac{dy}{dx} = e^{-2x} - 3y.
\end{align}
Using this slope, we calculate successive points along the curve using a small step size:
\begin{align}
x_1 &= x_0 + h, \quad y_1 = y_0 + h \frac{dy}{dx}\bigg|_{\brak{x_0, y_0}}.
\end{align}
For subsequent points:
\begin{align}
x_{n+1} &= x_n + h, \quad y_{n+1} = y_n + h\times\brak{e^{-2x_n} - 3y_n}.
\end{align}
Repeating this process for a large number of points, we can trace the curve that represents one of the solutions of the Differential Equation.
\end{frame}

\section{Figure}
\subsection{Figure Obtained}
\begin{frame}[fragile]
\frametitle{Figure Obtained}
\begin{figure}[ht!]
   \centering
   \includegraphics[width=0.7\linewidth]{Graph.jpeg}
    \label{stemplot}
\end{figure}
\end{frame}

\subsection{Python and C Codes for Simulation}
\begin{frame}[fragile]
\frametitle{Python and C Codes for Simulation}
The code in 
{\footnotesize
\begin{lstlisting}
https://github.com/ee24btech11054/1003/blob/main/Desktop/Eqdiff/codes/generate.c
\end{lstlisting}
}
generates the points which lie on the graph by using Difference Equation.

The code in 
{\footnotesize
\begin{lstlisting}
https://github.com/ee24btech11054/1003/blob/main/Desktop/Eqdiff/codes/eqdiff.py
\end{lstlisting}
}
plots the theoretical graph and the graph obtained by simulation in the plot.
\end{frame}

\end{document}



